%\documentclass[sigconf]{acmart}
%\acmConference[IFL'17]{International Symposium on Implementation and Application of Functional Languages}{August 2017}{Bristol, United Kingdom}
%\acmYear{2017}
%\copyrightyear{2017}

\documentclass[sigconf]{acmart}
\acmConference[IFL'17]{International Symposium on Implementation and Application of Functional Languages}{August 2017}{Bristol, United Kingdom}
\acmYear{2018}
\copyrightyear{2018}

%The ISBN number assigned to IFL 2017 is 978-1-4503-6343-3, and can be entered into the template as:

\acmISBN{978-1-4503-6343-3}


%\documentclass[sigplan,10pt]{acmart}

%\acmISBN{} % \acmISBN{978-x-xxxx-xxxx-x/YY/MM}
\acmDOI{} % \acmDOI{10.1145/nnnnnnn.nnnnnnn}
\startPage{1}

%% Copyright information
%% Supplied to authors (based on authors' rights management selection;
%% see authors.acm.org) by publisher for camera-ready submission;
%% use 'none' for review submission.
\setcopyright{none}
%\setcopyright{acmcopyright}
%\setcopyright{acmlicensed}
%\setcopyright{rightsretained}
%\copyrightyear{2017}           %% If different from \acmYear

\bibliographystyle{ACM-Reference-Format}
\usepackage{flushend} 
\usepackage{amssymb}
\setcounter{tocdepth}{3}
\usepackage[export]{adjustbox}
\usepackage{graphicx}
\usepackage{fancyvrb}

\usepackage{amsmath}
\usepackage{latexsym}
%\usepackage[usenames,dvipsnames]{color}

\usepackage{url}

\usepackage[T1]{fontenc}
\usepackage{lmodern}
\usepackage{amssymb,amsmath}
%\input{shadded}

\newcounter{figpart}[figure]
\renewcommand{\thefigpart}{}%%\alph{figpart}}

\usepackage[utf8]{inputenc}
\usepackage{enumitem}
\usepackage{hyperref}
\usepackage{bashful}
\usepackage{setspace}
\usepackage{color}
\usepackage{xcolor}
%% \lstloadlanguages{Erlang}
%% \lstset{
%%   frame=tb,
%%   rulecolor=\color{gray},
%%   basicstyle=\small\ttfamily\setstretch{0.8},
%%   breakindent=4ex,
%%   postbreak=(cont.),
%%   breaklines=true,
%%   showspaces=false,
%%   showtabs=false,
%%   xleftmargin=15pt,
%%   framexleftmargin=5pt,
%%   framexrightmargin=0pt,
%%   framexbottommargin=0pt,
%%   captionpos=b,
%%   numbers=left,
%%   numberstyle=\tiny\color{gray},
%% }
%% \hypersetup{
%%     colorlinks,
%%     linkcolor={red},
%%     citecolor={blue!50!black},
%%     urlcolor={blue!80!black}
%% }

\definecolor{codegreen}{rgb}{0,0.6,0}
\definecolor{codegray}{rgb}{0.5,0.5,0.5}
\definecolor{codeblue}{rgb}{0.0,0,0.62}
\definecolor{backcolour}{rgb}{0.95,0.95,0.95}

\newcommand{\lstbasicfont}{\small\fontfamily{pcr}\selectfont}
\newcommand{\lstcommentstyle}{\color{codegreen}\itshape}

\lstdefinestyle{mystyle}{
    backgroundcolor=\color{backcolour},
    commentstyle=\lstcommentstyle,
    keywordstyle=\color{codeblue}\bfseries,
    %identifierstyle=\color{black},
    identifierstyle=\color{codegray}\bfseries,
    %numberstyle=\tiny\color{codegray},
    stringstyle=\color{purple},
    basicstyle=\lstbasicfont,
    %basicstyle=\small\ttfamily,
    breakatwhitespace=false,
    breaklines=true,
    captionpos=b,
    keepspaces=true,
    %numbers=left,
    %numbersep=5pt,
    showspaces=false,
    showstringspaces=false,
    showtabs=false,
    tabsize=2
}

\lstdefinelanguage{myerlang}[]{erlang}{
  morekeywords={spec,fun,bitstring,orelse,record,include_lib},
  otherkeywords={|,||,<-,->,!,[,],\{,\}},
  deletekeywords={list,ok}
}

\lstset{style=mystyle,
  literate=
    {∪}{{$\cup$}}1 {∩}{{$\cap$}}1 {∈}{{$\in$}}1 {⊆}{{$\subseteq$}}1
    {∘}{{$\circ$}}1 {-}{{-}}1 {~}{{\texttildelow}}1 {<}{{<}}1
    {>}{{>}}1
  }

%\newcommand{\edge}[1]
%           {\stackrel{\bf{#1}}{\rightarrow}}
\newcommand{\edge}[1]
           {\xrightarrow{\bf{#1}}}
\newcommand{\rel}[1]
           {\stackrel{\bf{#1}}{\leadsto}}

\newcommand{\dc}{\textit{d\&c}}


\begin{document}

\title[Pattern Candidate Discovery and Parallelization Techniques]{Pattern Candidate Discovery and \\ Parallelization Techniques}

\author{Melinda T\'oth, Istv\'an Boz\'o, Tam\'as Kozsik}
%\authornote{with author1 note}
\orcid{nnnn-nnnn-nnnn-nnnn}
\affiliation{
%  \position{Position1}
%  \department{Department1}              %% \department is recommended
  \institution{ELTE, E\"otv\"os Lor\'and University}            %% \institution is required
%  \streetaddress{Street1 Address1}
  \city{Budapest}
%  \state{State1}
%  \postcode{Post-Code1}
  \country{Hungary}                    %% \country is recommended
}
\email{{tothmelinda,bozoistvan,kto}@elte.hu}          %% \email is recommended

%% \author{Melinda T\'oth, Istv\'an Boz\'o, Tam\'as Kozsik}


\begin{abstract}
Parallel computations in a program can be expressed conveniently,
at a high level of abstraction, using parallel patterns such as
task farm, pipeline or divide-and-conquer. In order to transform a sequential
program into a pattern-based parallel one, the software developer
may want to apply refactoring transformations on it. This tutorial
explains a methodology to perform tool supported parallelization of
programs by presenting how to use a specific static source code
analysis and transformation system for Erlang. A key element of the
approach is pattern candidate discovery, a static analysis technique
to identify code fragments that can be refactored into a parallel
pattern.
\end{abstract}

%% 2012 ACM Computing Classification System (CSS) concepts
%% Generate at 'http://dl.acm.org/ccs/ccs.cfm'.

\begin{CCSXML}
<ccs2012>
<concept>
<concept_id>10011007.10011006.10011073</concept_id>
<concept_desc>Software and its engineering~Software maintenance tools</concept_desc>
<concept_significance>500</concept_significance>
</concept>
<concept>
<concept_id>10011007.10011074.10011092</concept_id>
<concept_desc>Software and its engineering~Software development techniques</concept_desc>
<concept_significance>500</concept_significance>
</concept>
</ccs2012>
\end{CCSXML}

\ccsdesc[500]{Software and its engineering~Software maintenance tools}
\ccsdesc[500]{Software and its engineering~Software development techniques}

%% End of generated code


%% Keywords
%% comma separated list
\keywords{pattern based programming, Erlang, PaRTE, static analysis, refactoring}  %% \keywords are mandatory in final camera-ready submission


%% \maketitle
%% Note: \maketitle command must come after title commands, author
%% commands, abstract environment, Computing Classification System
%% environment and commands, and keywords command.
\maketitle

\section{Introduction}
\input{intro}

\section{Example: a pattern candidate}
Some people use the terms \emph{parallel} and \emph{concurrent}
interchangeably. Others prefer to separate these two concepts~\cite{marlow},
saying that

\begin{itemize}
\itemsep1pt\parskip0pt\parsep0pt
\item
  concurrency is when you define a system with a number of
  communicating, co-operating processes, typically exhibiting
  non-determinism;
\item
  parallelism is when you speed up a computation by performing it on a
  number of processing elements -- but otherwise you arrive at the same
  result as a sequential (and deterministic) computation.
\end{itemize}

We argue that the two concepts should be approached with different
programming techniques. Concurrency might be too low-level to achieve
parallelism, and may become a bottleneck in effective software
development. Let us explain this by using a parallel programming
example: we shall define a function implementing sparse matrix-vector
multiplication.

%\subsection{Example: sparse matrix-vector
%multiplication}\label{example-sparse-matrix-vector-multiplication}

A sparse matrix is a matrix with lots of zero elements. One
can efficiently represent such a matrix as a list of its non-zero
elements and their position. Similarly, a sparse vector can be
represented as a list of pairs, where each pair contains an index and a
(non-zero) value. Consider the following list, which contains three pairs.

\begin{figure}[htbp]
\centering
%\includegraphics[width=.28\textwidth]{sparse_vector.pdf}
\input{sparse_vector.latex}
\end{figure}

This structure can be written as an Erlang \emph{term} 
using the \emph{list} constructor \lstinline[language=myerlang]{[]} and 
the \emph{tuple} constructor \lstinline[language=myerlang]|{}| as
like this: %\linebreak
\lstinline[language=myerlang]|[{3,15}, {7,31}, {8,12}]|.
Assuming 0-based indices, the above list can represent the following
vector (of length 10).

\begin{figure}[htbp]
\centering
%\includegraphics[width=.45\textwidth]{sparse_vector_full.pdf}
\input{sparse_vector_full.latex}
\end{figure}

Note that the list of pairs is sorted by the first (i.e. the index)
component --  this will be very
helpful for implementing operations, in our case the multiplication.
First of all, let us see how to multiply two sparse vectors $r$ and $c$
to compute their dot-product, $\sum r_i c_i$. Note that we need to sum
up $r_i c_i$ for all $i$ indices such that $r_i \neq 0$ and
$c_i \neq 0$.

\begin{figure*}[tbh]
\centering
%\includegraphics[width=.9\textwidth]{sparse_matrix.pdf}
\scalebox{0.8}{\input{sparse_matrix.latex}}
\caption{Sparse matrix representation}
\label{fig1}
\end{figure*}

The \lstinline[language=myerlang]{vxv/2} (that is \emph{vector-times-vector}) function\footnote{
Erlang supports overloading for functions, i.e. we can have functions with
the same name, but with different arity. To distinguish between overloaded
functions, the \emph{name/arity} notation is used, like in \lstinline[language=myerlang]{vxv/2}.
} takes
two lists that represent the two sparse vectors, and computes the
dot-product recursively, by traversing the two lists simultaneously.
When both vectors contain a non-zero value at a given index position
$i$, they are multiplied, and the product is added to the final result
-- as described by the first \emph{clause} of the following definition.

\begin{lstlisting}[language=myerlang]
vxv( [{I,R} | Rs], [{I,C} | Cs] ) ->
  R*C + vxv(Rs,Cs);
vxv( [{I,R} | Rs], [{J,C} | Cs] ) ->
  if
    I<J  -> vxv( Rs, [{J,C}|Cs]);
    true -> vxv( [{I,R} | Rs], Cs )
  end;
vxv( _, _ ) ->    % when either (or both)
  0.              % of the lists is empty

\end{lstlisting}

%% \begin{Shaded}
%% \begin{Highlighting}[]
%% \FunctionTok{vxv(} \FunctionTok{[\{}\DataTypeTok{I}\FunctionTok{,}\DataTypeTok{R}\FunctionTok{\}} \FunctionTok{|} \DataTypeTok{Rs}\FunctionTok{],} \FunctionTok{[\{}\DataTypeTok{I}\FunctionTok{,}\DataTypeTok{C}\FunctionTok{\}} \FunctionTok{|} \DataTypeTok{Cs}\FunctionTok{]} \FunctionTok{)} \KeywordTok{->} \DataTypeTok{R}\KeywordTok{*}\DataTypeTok{C} \KeywordTok{+} \FunctionTok{vxv(}\DataTypeTok{Rs}\FunctionTok{,}\DataTypeTok{Cs}\FunctionTok{);}

%% \FunctionTok{vxv(} \FunctionTok{[\{}\DataTypeTok{I}\FunctionTok{,}\DataTypeTok{R}\FunctionTok{\}} \FunctionTok{|} \DataTypeTok{Rs}\FunctionTok{],} \FunctionTok{[\{}\DataTypeTok{J}\FunctionTok{,}\DataTypeTok{C}\FunctionTok{\}} \FunctionTok{|} \DataTypeTok{Cs}\FunctionTok{]} \FunctionTok{)} \KeywordTok{->}
%%     \KeywordTok{if}
%%         \DataTypeTok{I} \KeywordTok{<} \DataTypeTok{J} \KeywordTok{->} \FunctionTok{vxv(}          \DataTypeTok{Rs}\FunctionTok{,}  \FunctionTok{[\{}\DataTypeTok{J}\FunctionTok{,}\DataTypeTok{C}\FunctionTok{\}} \FunctionTok{|} \DataTypeTok{Cs}\FunctionTok{]} \FunctionTok{);}
%%         \CharTok{true}  \KeywordTok{->} \FunctionTok{vxv(} \FunctionTok{[\{}\DataTypeTok{I}\FunctionTok{,}\DataTypeTok{R}\FunctionTok{\}} \FunctionTok{|} \DataTypeTok{Rs}\FunctionTok{],}          \DataTypeTok{Cs}  \FunctionTok{)}
%%     \KeywordTok{end}\FunctionTok{;}

%% \FunctionTok{vxv(} \DataTypeTok{_}\FunctionTok{,} \DataTypeTok{_} \FunctionTok{)} \KeywordTok{->} \DecValTok{0}\FunctionTok{.}    \CommentTok{% when either of the lists is empty}
%% \end{Highlighting}
%% \end{Shaded}

The second clause of the function definition tells us what to do when
the head of the two input lists corresponds to non-empty vector elements
\lstinline[language=myerlang]{R} and \lstinline[language=myerlang]{C} at different index positions (\lstinline[language=myerlang]{I} and
\lstinline[language=myerlang]{J}, respectively). Now we make use of the fact that the lists
are sorted by the index component. If \lstinline[language=myerlang]{I < J}, the
non-zero elements stored in the right list are all at an index position
larger than \lstinline[language=myerlang]{I}, and hence \lstinline[language=myerlang]{R} is not needed for the
dot-product: this corresponds to $r_i c_i$ with $r_i \neq 0$ and
$c_i = 0$ in the sum. Therefore, we discard the head of the left list,
and proceed recursively. Similarly, when \lstinline[language=myerlang]{I > J}, we
discard the head of the right list, and proceed recursively.

The last clause of \lstinline[language=myerlang]{vxv/2} is reached when either of the lists is
empty. In this case, the corresponding sparse vector does not contain
any more non-zero elements, and hence the dot-product is $0$.

%Okay, you can easily refactor the definition above into a tail-recursive
%one by introducing an accumulator. This can improve performance for long
%lists.
%
%\begin{Shaded}
%\begin{Highlighting}[]
%\FunctionTok{vxv(}\DataTypeTok{Row}\FunctionTok{,}\DataTypeTok{Col}\FunctionTok{)} \KeywordTok{->} \FunctionTok{vxv_acc(}\DataTypeTok{Row}\FunctionTok{,}\DataTypeTok{Col}\FunctionTok{,}\DecValTok{0}\FunctionTok{).}
%
%\FunctionTok{vxv_acc(} \FunctionTok{[\{}\DataTypeTok{I}\FunctionTok{,}\DataTypeTok{R}\FunctionTok{\}|}\DataTypeTok{Rs}\FunctionTok{],} \FunctionTok{[\{}\DataTypeTok{I}\FunctionTok{,}\DataTypeTok{C}\FunctionTok{\}|}\DataTypeTok{Cs}\FunctionTok{],} \DataTypeTok{Acc} \FunctionTok{)} \KeywordTok{->} \FunctionTok{vxv_acc(}\DataTypeTok{Rs}\FunctionTok{,} \DataTypeTok{Cs}\FunctionTok{,} \DataTypeTok{R}\KeywordTok{*}\DataTypeTok{C} \KeywordTok{+} \DataTypeTok{Acc}\FunctionTok{);}
%
%\FunctionTok{vxv_acc(} \FunctionTok{[\{}\DataTypeTok{I}\FunctionTok{,}\DataTypeTok{R}\FunctionTok{\}|}\DataTypeTok{Rs}\FunctionTok{],} \FunctionTok{[\{}\DataTypeTok{J}\FunctionTok{,}\DataTypeTok{C}\FunctionTok{\}|}\DataTypeTok{Cs}\FunctionTok{],} \DataTypeTok{Acc} \FunctionTok{)} \KeywordTok{->}
%    \KeywordTok{if}
%       \DataTypeTok{I} \KeywordTok{<} \DataTypeTok{J} \KeywordTok{->} \FunctionTok{vxv_acc(}        \DataTypeTok{Rs}\FunctionTok{,}  \FunctionTok{[\{}\DataTypeTok{Y}\FunctionTok{,}\DataTypeTok{C}\FunctionTok{\}|}\DataTypeTok{Cs}\FunctionTok{],} \DataTypeTok{Acc} \FunctionTok{);}
%       \CharTok{true}  \KeywordTok{->} \FunctionTok{vxv_acc(} \FunctionTok{[\{}\DataTypeTok{I}\FunctionTok{,}\DataTypeTok{R}\FunctionTok{\}|}\DataTypeTok{Rs}\FunctionTok{],}        \DataTypeTok{Cs}\FunctionTok{,}  \DataTypeTok{Acc} \FunctionTok{)}
%    \KeywordTok{end}\FunctionTok{;}
%
%\FunctionTok{vxv_acc(} \DataTypeTok{_}\FunctionTok{,} \DataTypeTok{_}\FunctionTok{,} \DataTypeTok{Acc} \FunctionTok{)} \KeywordTok{->} \DataTypeTok{Acc}\FunctionTok{.}   \CommentTok{% when either of the lists is empty}
%\end{Highlighting}
%\end{Shaded}

Now we are ready to implement the matrix-vector multiplication, the
\lstinline[language=myerlang]{mxv/2} function. A sparse matrix is represented as a list of
those rows that contain at least one non-zero element. Each such row in
the matrix is represented as the pair of the row index, and a
(non-empty) list representing a sparse vector (Figure~\ref{fig1}).

The product of a matrix and a (column) vector is a (row) vector, the
elements of which are the dot-product of a row vector in the matrix and
the given column vector. The row vector is stored in variable
\lstinline[language=myerlang]{DotProducts} (in Erlang the name of a variable must start with a capital
letter, while the name of a function must start with a lowercase letter).
The last step of the computation is to drop zero values from \lstinline[language=myerlang]{DotProducts}
using the \lstinline[language=myerlang]{filter/2} function from the
standard library \emph{module} \lstinline[language=myerlang]{lists}.

\begin{lstlisting}[language=myerlang]
mxv( Rows, Col ) ->
  DotProducts =
    [ {I,vxv(Row,Col)} || {I,Row} <- Rows ],
  lists:filter( fun({_I,V}) -> V /= 0 end,
                DotProducts).
\end{lstlisting}

The \lstinline[language=myerlang]{lists:filter/2} function is a higher-order function: its first
argument is a predicate, in this case an \emph{explicit fun-expression}
(often called as lambda-expression in other languages). This specific predicate
takes a pair, discards its first element (we should start the name of
unused parameters/variables with an underscore), and returns true 
\emph{iff} the second element is non-zero.


%% \begin{Shaded}
%% \begin{Highlighting}[]
%% \FunctionTok{mxv(} \DataTypeTok{Rows}\FunctionTok{,} \DataTypeTok{Col} \FunctionTok{)} \KeywordTok{->}
%%     \DataTypeTok{DotProducts} \KeywordTok{=} \FunctionTok{[} \FunctionTok{\{}\DataTypeTok{I}\FunctionTok{,vxv(}\DataTypeTok{Row}\FunctionTok{,}\DataTypeTok{Col}\FunctionTok{)\}} \FunctionTok{||} \FunctionTok{\{}\DataTypeTok{I}\FunctionTok{,}\DataTypeTok{Row}\FunctionTok{\}} \KeywordTok{<-} \DataTypeTok{Rows} \FunctionTok{],}
%%     \FunctionTok{lists:filter(} \KeywordTok{fun}\FunctionTok{(\{}\DataTypeTok{_I}\FunctionTok{,}\DataTypeTok{V}\FunctionTok{\})} \KeywordTok{->} \DataTypeTok{V} \KeywordTok{/=} \DecValTok{0} \KeywordTok{end}\FunctionTok{,} \DataTypeTok{DotProducts} \FunctionTok{).}
%% \end{Highlighting}
%% \end{Shaded}

\subsection{Compute faster with
processes!}\label{compute-faster-with-processes}

It is very easy to speed up the computation of \lstinline[language=myerlang]{mxv/2} by
spawning processes, and let them calculate the dot-products in parallel.
%We already know how to use the concurrency features of Erlang to spawn
%processes, as well as to send and receive messages. We can easily
One can easily
rewrite \lstinline[language=myerlang]{mxv/2} to spawn a process for each represented row in
the matrix, and instruct those processes to calculate the dot-products
using \lstinline[language=myerlang]{vxv_proc/3} (a wrapper around \lstinline[language=myerlang]{vxv/2}). In the rewritten
\lstinline[language=myerlang]{mxv/2} function, we bind the list of row indices (\lstinline[language=myerlang]{I}) and
the corresponding process identifiers (\lstinline[language=myerlang]{Pid}) to the variable
\lstinline[language=myerlang]{IndicesAndPids}. The dot-products are communicated back from the
processes and collected by the \lstinline[language=myerlang]{receive} expression in the
spawning order of the processes, and hence in the order of row indices.

\begin{lstlisting}[language=myerlang]
mxv( Rows, Col ) ->
  IndicesAndPids =
    [ {I, spawn(?MODULE, vxv_proc,
                [self(), Row, Col]
      || {I,Row} <- Rows ],
  DotProducts =
    [ receive
        {Pid, Res} -> {I, Res}
      end
      || {I, Pid} <- IndicesAndPids ],
  lists:filter( fun({_,V}) -> V /= 0 end,
                DotProducts ).
\end{lstlisting}

%% \begin{Shaded}
%% \begin{Highlighting}[]
%% \FunctionTok{mxv(} \DataTypeTok{Rows}\FunctionTok{,} \DataTypeTok{Col} \FunctionTok{)} \KeywordTok{->}
%%     \DataTypeTok{IndicesAndPids} \KeywordTok{=} 
%%         \FunctionTok{[ \{}\DataTypeTok{I}\FunctionTok{,} \FunctionTok{spawn(?}\DataTypeTok{MODULE}\FunctionTok{,} \CharTok{vxv_proc}\FunctionTok{,} \FunctionTok{[self(),} \DataTypeTok{Row}\FunctionTok{,} \DataTypeTok{Col}\FunctionTok{])\}}
%%         \FunctionTok{||} \FunctionTok{\{}\DataTypeTok{I}\FunctionTok{,}\DataTypeTok{Row}\FunctionTok{\}} \KeywordTok{<-} \DataTypeTok{Rows}\FunctionTok{ ],}
%%     \DataTypeTok{DotProducts} \KeywordTok{=} \FunctionTok{[} \KeywordTok{receive} 
%%                       \FunctionTok{\{}\DataTypeTok{Pid}\FunctionTok{,} \DataTypeTok{Res}\FunctionTok{\}} \KeywordTok{->} \FunctionTok{\{}\DataTypeTok{I}\FunctionTok{,} \DataTypeTok{Res}\FunctionTok{\}}
%%                     \KeywordTok{end} 
%%                   \FunctionTok{||} \FunctionTok{\{}\DataTypeTok{I}\FunctionTok{,} \DataTypeTok{Pid}\FunctionTok{\}} \KeywordTok{<-} \DataTypeTok{IndicesAndPids} \FunctionTok{],}
%%     \FunctionTok{lists:filter(} \KeywordTok{fun}\FunctionTok{(\{}\DataTypeTok{_}\FunctionTok{,}\DataTypeTok{V}\FunctionTok{\})} \KeywordTok{->} \DataTypeTok{V} \KeywordTok{/=} \DecValTok{0} \KeywordTok{end}\FunctionTok{,} \DataTypeTok{DotProducts} \FunctionTok{).}
%% \end{Highlighting}
%% \end{Shaded}

Dot-products are computed in parallel by processes that execute
\lstinline[language=myerlang]{vxv_proc/3}. When a new process is spawned, we pass the process
identifier of the ``master'' process, \lstinline[language=myerlang]{self()}, together with the
row vector and the column vector to this function. This
\lstinline[language=myerlang]{vxv_proc/3} simply calls \lstinline[language=myerlang]{vxv/2}, and sends back the
computed dot-product (together with the process id of the ``slave''
process, \lstinline[language=myerlang]{self()}) to the parent process.

\begin{lstlisting}[language=myerlang]
vxv_proc(ParentPid, Row, Col) ->
  ParentPid ! {self(), vxv(Row, Col)}.
\end{lstlisting}

%% \begin{Shaded}
%% \begin{Highlighting}[]
%% \FunctionTok{vxv_proc(}\DataTypeTok{ParentPid}\FunctionTok{,} \DataTypeTok{Row}\FunctionTok{,} \DataTypeTok{Col}\FunctionTok{)} \KeywordTok{->}
%%     \DataTypeTok{ParentPid} \KeywordTok{!} \FunctionTok{\{self(),} \FunctionTok{vxv(}\DataTypeTok{Row}\FunctionTok{,} \DataTypeTok{Col}\FunctionTok{)\}.}
%% \end{Highlighting}
%% \end{Shaded}

The concurrency features of Erlang allowed us to parallelize the sparse
matrix-vector multiplication by introducing processes, spawning them,
and implementing communication among them based on the process
identifiers. One may consider this a relatively small overhead to arrive
at a parallel computation. However, there are a number of
% ways here to shoot ourselves in the foot. 
small stupid mistakes one can make.
For example, consider the two calls to
\lstinline[language=myerlang]{self/0}. This impure function returns the \emph{pid} of the
process that evaluates the call. The two calls to \lstinline[language=myerlang]{self/0} will
be evaluated by different processes here: the one in \lstinline[language=myerlang]{mxv/2} is
evaluated in the master process, while the other one in the spawned
processes. % It is quite easy to get confused\ldots{}
The lack of referential transparency can be rather confusing in this
case.

Another opportunity for shooting yourself on the foot is the following.
One could easily come to the idea that
the definition of \lstinline[language=myerlang]{mxv/2} can be refactored, and the two
variables can be eliminated. These variables are used only once, so why
not replace them with their definition?

\begin{lstlisting}[language=myerlang]
mxv( Rows, Col ) ->
  lists:filter(
    fun({_,V}) ->
      V /= 0 end,
      [ receive
          {Pid, Res} -> {I, Res}
        end
        || {I, Row} <- Rows,
           Pid <- [spawn(?MODULE, vxv_proc,
                         [self(),Row,Col])]
      ]
    ).        
\end{lstlisting}

%% \begin{Shaded}
%% \begin{Highlighting}[]
%% \FunctionTok{mxv(} \DataTypeTok{Rows}\FunctionTok{,} \DataTypeTok{Col} \FunctionTok{)} \KeywordTok{->}
%%     \FunctionTok{lists:filter(} 
%%          \KeywordTok{fun}\FunctionTok{(\{}\DataTypeTok{_}\FunctionTok{,}\DataTypeTok{V}\FunctionTok{\})} \KeywordTok{->} \DataTypeTok{V} \KeywordTok{/=} \DecValTok{0} \KeywordTok{end}\FunctionTok{,}
%%          \FunctionTok{[} \KeywordTok{receive} 
%%              \FunctionTok{\{}\DataTypeTok{Pid}\FunctionTok{,} \DataTypeTok{Res}\FunctionTok{\}} \KeywordTok{->} \FunctionTok{\{}\DataTypeTok{I}\FunctionTok{,} \DataTypeTok{Res}\FunctionTok{\}}
%%            \KeywordTok{end} 
%%          \FunctionTok{||} \FunctionTok{\{}\DataTypeTok{I}\FunctionTok{,} \DataTypeTok{Row}\FunctionTok{\}} \KeywordTok{<-} \DataTypeTok{Rows} \FunctionTok{,}
%%             \DataTypeTok{Pid} \KeywordTok{<-} \FunctionTok{[spawn(?}\DataTypeTok{MODULE}\FunctionTok{,} \CharTok{vxv_proc}\FunctionTok{,} \FunctionTok{[self(),} \DataTypeTok{Row}\FunctionTok{,} \DataTypeTok{Col}\FunctionTok{])]}
%%          \FunctionTok{]}
%%     \FunctionTok{).}
%% \end{Highlighting}
%% \end{Shaded}

Of course, it is a matter of taste whether we consider this new
definition simpler or more obscure -- but this refactoring might seem
reasonable at first glance. However, it turns out that
there is no parallel computation in this solution any more. 
It is true that the list
generator in the second argument of \lstinline[language=myerlang]{lists:filter/2} starts a new
process for each row, and receives the dot-product of this row and
\lstinline[language=myerlang]{Col} from this process. But note that the next process is not spawned
until the result from the previous one is received. There is an
unnecessary synchronization in this definition that prevents parallel
execution. The refactoring we applied was valid in the sense that this
new definition produces exactly the same result as the previous one,
hence the definitions can be regarded as semantically equivalent.
But obviously, losing parallelism was not intended!

\subsection{Simply parallel}\label{simply-parallel}

Now we can see that implementing parallel computations using the
concurrency features of a language can be error-prone, and sometimes
really cumbersome. In the following we will see how to write a parallel
program in a way that avoids this low-level trickery with concurrency,
synchronization, communication, process management and so on. This new
technique will allow us to describe a parallel computation at a high
level of abstraction, in terms of common solution recipes,
\emph{parallel patterns}.
Intuitively, the structure of the parallel computation for
our sparse matrix-vector multiplication, \lstinline[language=myerlang]{Computation}, can be
given as a task farm with \lstinline[language=myerlang]{length(Rows)} tasks, each task
performing a ``sequential'' component \lstinline[language=myerlang]{Function}, and forcing an
ordered collection of results. The computational structure is described
using \emph{algorithmic skeletons} (\lstinline[language=myerlang]{ord}, \lstinline[language=myerlang]{farm},
\lstinline[language=myerlang]{seq}). The computation is carried out by the \lstinline[language=myerlang]{skel:do/2}
function, the main entry point to the Skel library.

\begin{lstlisting}[language=myerlang]
mxv( Rows, Col ) ->
  Function =
    fun ({I,Row}) -> {I,vxv(Row,Col)} end,
  Computation =
    [{ord,[{farm,[{seq,Function}],
            length(Rows)}]}],
  DotProducts = skel:do( Computation, Rows),
  lists:filter( fun({_I,V}) -> V /= 0 end,
                DotProducts).
\end{lstlisting}

%% \begin{Shaded}
%% \begin{Highlighting}[]
%% \FunctionTok{mxv(} \DataTypeTok{Rows}\FunctionTok{,} \DataTypeTok{Col} \FunctionTok{)} \KeywordTok{->}
%%     \DataTypeTok{Function} \KeywordTok{=} \KeywordTok{fun} \FunctionTok{(\{}\DataTypeTok{I}\FunctionTok{,}\DataTypeTok{Row}\FunctionTok{\})} \KeywordTok{->} \FunctionTok{\{}\DataTypeTok{I}\FunctionTok{,vxv(}\DataTypeTok{Row}\FunctionTok{,}\DataTypeTok{Col}\FunctionTok{)\}} \KeywordTok{end}\FunctionTok{,}
%%     \DataTypeTok{Computation} \KeywordTok{=} \FunctionTok{[\{}\CharTok{ord}\FunctionTok{,[\{}\CharTok{farm}\FunctionTok{,[\{}\CharTok{seq}\FunctionTok{,}\DataTypeTok{Function}\FunctionTok{\}],length(}\DataTypeTok{Rows}\FunctionTok{)\}]\}],}
%%     \DataTypeTok{DotProducts} \KeywordTok{=} \FunctionTok{skel:do(} \DataTypeTok{Computation}\FunctionTok{,} \DataTypeTok{Rows} \FunctionTok{),}
%%     \FunctionTok{lists:filter(} \KeywordTok{fun}\FunctionTok{(\{}\DataTypeTok{_I}\FunctionTok{,}\DataTypeTok{V}\FunctionTok{\})} \KeywordTok{->} \DataTypeTok{V} \KeywordTok{/=} \DecValTok{0} \KeywordTok{end}\FunctionTok{,} \DataTypeTok{DotProducts} \FunctionTok{).}
%% \end{Highlighting}
%% \end{Shaded}

In the forthcoming sections we cover some details of parallel patterns,
and discover how Skel, an algorithmic skeleton library for Erlang 
allows us to express the structure of parallel
computations. %Finally, we shall see how to parallelize existing programs
%by refactoring them, and introducing parallel patterns in their code.



\input{patterns}
%\input{skel}


\section{Tool support for parallelisation}

\input{parte-tool}

\section{How to build a tool like PaRTE?}
\input{analysis}


\section{Evaluation}
\input{eval}

\section{Related work}
\input{related}


\section{Summary}

The ParaPhrase Refactoring Tool for Erlang facilitates the parallelization
of Erlang source code by providing automated pattern candidate discovery and
refactoring transformations. This paper presents the underlying methodology,
and teaches the use of the tool. Moreover, we explain how the static source
code analyses work for parallel pattern candidate discovery -- this information
can be used to apply the methodology, and implement the parallelization tool,
in the context of other programming languages. Currently we are working on
a refactoring tool-chain for Scala. This language has a concurrency tool-set (Akka)
very similar to that of Erlang. Therefore, the main ideas presented in this
paper can be reused in the parallelisation of Scala programs -- though many interesting
problems remain due to the differences in the two languages. Even more
problems can arise when targeting other programming languages (e.g. OpenMP
would be a logical choice).

This presentation focused on a set of well-known parallel patterns, including
the task farm, the pipeline and the divide-and-conquer patterns. We explained
what kind of program fragments (pattern candidates) can be automatically
identified by PaRTE, and how to select the most promising candidates.
The selected candidates can be refactored into
applications of an algorithmic skeleton library, which implements the
parallel patterns. Many of the refactoring transformations can be performed
automatically by the tool, though the most complex ones can only be carried
out by hand currently. We have evaluated the applicability of the methodology
on a number of real-world, open source code bases, and found that the tool
is indeed able to identify many interesting parallel pattern candidates.

There are still, of course, some shortcomings and unsolved implementation
issues. We do not claim that our parallel pattern candidate discovery is
either complete or even sound. Some of the discovery rules are trade-offs
between soundness and (efficient) computability. It might happen that the
tool presents a ``fake candidate'', which cannot be refactored into a pattern.
This should not be a problem, however, since our tool is not a compiler
optimization running unsupervised. Fake candidates can be discarded by the
user after proper consideration.

The completeness of the discovery is not guaranteed either. Our rules can
find many interesting candidates, but there may be many more: in the future
the rule set can be extended, or can be made more accurate. What is even
more important is the good ranking of the candidates. Completely automatic
performance estimations are not feasible in more complex cases, so either
good heuristics are required (such as our existing one using some complexity
metrics), or the integration  of our tool with an environment supporting
development-time experimentation and dynamic analyses.

Automatic transformations also have a number of unsolved issues. It might
happen that a candidate is found by the discovery, but the current
definition of the appropriate transformation is not able to cover that
specific candidate. In this case either manual transformation of the code
is needed (which is error-prone) or the application of a sequence of
built-in transformations including program shaping before, and code clean-up
after, the introduction of the desired pattern (like in the case of the \dc{}
pattern). We are working actively to ensure, and also to prove, soundness
of the refactoring transformations -- clearly, this is a crucial requirement
for our approach.


\subsection*{Acknowledgement}
The research has been supported by the European Union, co-financed by the European Social Fund (EFOP-3.6.2-16-2017-00013) in 2017-2018, 
by the Hungarian Government through the New National Excellence Program of the Ministry of Human Capacities in 2016-2017, 
by the Higher Education Restructuring Fund of the Hungarian Government in 2015-2016, 
and by the Seventh Framework Programme (FP7) of the European Union under the contract number 288570 in 2013-2015.

\bibliography{ifl17}

%% \appendix
%% \setcounter{lemma}{0}
%% \section{The detailed proofs of lemmas.}\label{proofs}
%% \input{lemmas.tex}

\end{document}
